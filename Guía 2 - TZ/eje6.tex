\chapter*{Ejercicio 6}
\addcontentsline{toc}{chapter}{Ejercicio 6}

\section{Parte A}

Sabemos que $H\left( e^{jw} \right)=H\left( Z \right)|_{z=e^{jw}}$, es solo un cambio de variable. También sabemos que $\left| H\left( e^{jw} \right) \right|^2= H\left( e^{-jw} \right) H\left( e^{jw} \right)$, por lo tanto podemos realizar las siguientes igualdades:\\
$\left| H\left( e^{jw} \right) \right|^2=H\left( e^{-jw} \right) H\left( e^{jw} \right)=H\left( Z \right) H\left( Z^{-1} \right)|_{z=e^{jw}}$
Quedando demostrado que $\left| H\left( e^{jw} \right) \right|^2=H\left( Z \right) H\left( Z^{-1} \right)|_{z=e^{jw}}$

\section{Parte B}
Teniendo que el filtro es:
$$H\left( Z \right)=\frac{1-aZ+bZ^2}{b-aZ+Z^2}$$
Para ver su comportamiento en todas las frecuencias realizamos:
$$H\left( Z \right)H\left( Z^{-1} \right)=\frac{1-aZ+bZ^2}{b-aZ+Z^2}* \frac{1-aZ^{-1}+bZ^{-2}}{b-aZ^{-1}+Z^{-2}}=\frac{1-aZ+bZ^2}{b-aZ+Z^2}* \frac{Z^{2}-aZ+b}{bZ^{2}-aZ+1}=1$$
Y como establecimos anteriormente, $\left| H\left( e^{jw} \right) \right|^2=H\left( Z \right) H\left( Z^{-1} \right)|_{z=e^{jw}}$, podemos observar que es un filtro pasa todo.

\section{Parte C}
Dado que los polos de un filtro de 2° orden son dados por el polinomio $b-aZ+Z^2$ entonces estos se van a encontrar definidos como $\frac{a \pm \sqrt{a^2+ 4b}}{2}$. De esta manera podemos observar que estos van a ser pares conjugados. Sabemos que podemos definir los polos de un orden m por un polinomio genérico, esté puede definirse de la siguiente manera: 
$$\left( b_0-a_0Z+Z^2 \right)*\left( b_1-a_1Z+Z^2 \right)*\left( b_2-a_2Z+Z^2 \right)\dots$$\\
En otras palabras, es una multiplicación de polinomios más chicos.\\
De igual manera, los ceros también pueden definirse como una multiplicación de polinomios más chicos. Entonces, tanto los polos como los ceros van a ocurrir en pares conjugados.\\
Pero previamente demostramos  que la respuesta en frecuencia de un filtro es $H\left( Z \right) H\left( Z^{-1} \right)|_{z=e^{jw}}$, luego vamos a tener la $H\left( Z^{-1} \right)$ donde sus polos van a ser $bZ^2-aZ+1$ y con los ceros va a suceder lo mismo. Por lo tanto, los ceros y los polos van a ocurrir en pares recíprocos.\\
Finalmente,podemos concluir que tanto los polos como los ceros van a suceder en pares conjugados y recíprocos.