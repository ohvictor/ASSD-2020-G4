\chapter{Filtro Pasa Bajos}
Se diseñaron filtros pasa bajos para funcionar como el Filtro Anti Alias (FAA) y el Filtro Recuperador (FR) en el esquema a implementar.
\section{Especificaciones del Filtro}
Se pidió que el Filtro Pasa Bajos cumpla con las siguientes especificaciones:

\begin{table}[ht]
    \centering
    \begin{tabular}{|c|c|c|}
        \hline
        $f_a$  &   $A_p$   &   $A_a$    \\
        \hline
        $1.5 f_p$   &   $1 \si{\deci\bel}$ &   $41 \si{\deci\bel}$    \\
        \hline
    \end{tabular}
    \caption{Especificaciones de los Filtros Pasa Bajo}
\end{table}

Se eligió un $fp = 50 \si{\kilo\hertz}$ dado que las señales que serán muestreadas son 

\section{Realización del Filtro}

Para implementar el filtro pasa bajos se utilizaron celdas Rauch de construcción pasa bajos. Su esquema se muestra en la Figura \ref{fig:Rauch-Cell}.

\begin{figure}[ht]
    \centering
    \begin{circuitikz}[american voltages]
\draw
(0,0) node[op amp] (opamp) {}
(opamp.+) node[ground]{}
(opamp.-) to[resistor=$R_2$,*-*] ++(-3,0) coordinate(tmp)
(tmp) to[capacitor=$C_1$] ++(0,-2) node[ground]{}
(tmp) to[resistor=$R_1$,-o] ++(-3,0) node[left]{$V_{in}$}
(opamp.out) to[short] ++(0,2) coordinate(tmp1) to [short] ++(0,1) coordinate(tmp2)
(tmp1) to[capacitor=$C_2$] ++(-2,0) -| (opamp.-)
(tmp2) to[resistor,l_=$R_3$] ++(-2,0) -| (tmp)
(opamp.out) to[short,-o] ++(1,0) node[right]{$V_{out}$}
;
\end{circuitikz}
    \caption{Celda Rauch}
    \label{fig:Rauch-Cell}
\end{figure}



