\chapter{Filtro Pasa Bajos}
Se diseñaron filtros pasa bajos para funcionar como el Filtro Anti Alias (FAA) y el Filtro Recuperador (FR) en el esquema a implementar.
\section{Especificaciones del Filtro}
Se pidió que el Filtro Pasa Bajos cumpla con las siguientes especificaciones:

\begin{table}[ht]
    \centering
    \begin{tabular}{|c|c|c|}
        \hline
        $f_a$  &   $A_p$   &   $A_a$    \\
        \hline
        $1.5 f_p$   &   $1 \si{\deci\bel}$ &   $41 \si{\deci\bel}$    \\
        \hline
    \end{tabular}
    \caption{Especificaciones de los Filtros Pasa Bajo}
\end{table}

Se eligió un $fp = 50 \si{\kilo\hertz}$ dado que las señales que serán muestreadas son 

\section{Realización del Filtro}

Para implementar el filtro pasa bajos se utilizaron celdas Rauch de construcción pasa bajos. Su esquema se muestra en la Figura \ref{fig:Rauch-Cell}.

\begin{figure}[ht]
    \centering
    \begin{circuitikz}[american voltages]
\draw
(0,0) node[op amp] (opamp) {}
(opamp.+) node[ground]{}
(opamp.-) to[resistor=$R_2$,*-*] ++(-3,0) coordinate(tmp)
(tmp) to[capacitor=$C_1$] ++(0,-2) node[ground]{}
(tmp) to[resistor=$R_1$,-o] ++(-3,0) node[left]{$V_{in}$}
(opamp.out) to[short] ++(0,2) coordinate(tmp1) to [short] ++(0,1) coordinate(tmp2)
(tmp1) to[capacitor=$C_2$] ++(-2,0) -| (opamp.-)
(tmp2) to[resistor,l_=$R_3$] ++(-2,0) -| (tmp)
(opamp.out) to[short,-o] ++(1,0) node[right]{$V_{out}$}
;
\end{circuitikz}
    \caption{Celda Rauch}
    \label{fig:Rauch-Cell}
\end{figure}

La transferencia total de esta celda estará dada por la expresión:
\begin{equation}
    H(s)=\frac{R_3}{R_1}\cdot\frac{1/R_3 R_2 C_1 C_2}{s^2+s \frac{1}{C_1}(1/R_1+1/R_2+1/R_3)+1/R_2 R_3 C_1 C_2}
\end{equation}

Comparando con la expresión general de la transferencia de un filtro pasabajos de segundo orden:

\begin{equation*}
    H(s)=\frac{K \omega_0^2}{s^2 +s\frac{\omega_0}{Q_0}+\omega_0 ^2}
\end{equation*}

Se obtiene que:
\begin{align}
    K   &=  \frac{R_3}{R_1}\\
    \omega_0^2  &=  \frac{1}{R_2 R_3 C_1 C_2}\\
    \frac{\omega_0}{Q_0}    &=  \frac{1}{C_1}\left(\frac{1}{R_1}+\frac{1}{R_2}+\frac{1}{R_3}\right)
\end{align}

Dado que se desea que la ganancia en continua sea $H(0)=0\si{\deci\bel}$, se puede tomar $R_3 = R_1 = R$. Se definió entonces también $R_2=a^2R$ y $C_1 = k^2 C_2 = k^2 C$. Por lo tanto, la transferencia de la celda resulta en:

\begin{align}
    K&=1\\
    \omega_0^2&=\frac{1}{a^2 R^2 k^2 C^2} \Rightarrow \omega =\frac{1}{ak}\cdot\frac{1}{R C}\\
    \frac{\omega_0}{Q_0} &= \frac{1}{k^2 C} \left(\frac{2}{R}+\frac{1}{a^2 R}\right) \label{eq:filterwo/Q0}
\end{align}

Operando sobre \eqref{eq:filterwo/Q0} se obtiene

\begin{align*}
    \frac{1}{Q_0} &= \frac{1}{k^2 a^2 RC}\left(2 a^2 + 1 \right)\cdot kaRC\\
    \frac{1}{Q_0} &= \frac{1}{ka} \left(2a^2+1\right)\\
    Q_0 &= \frac{ka}{2a^2+1}
\end{align*}

Por lo tanto la expresión del factor de calidad de la celda $Q_0$ estará dada por
\begin{equation}
    Q_0 = \frac{ka}{2a^2+1}
\end{equation}